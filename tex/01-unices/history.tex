\documentclass[../ops.tex]{subfiles}

\begin{document}
        \subsection{Genesis}
        In the beginning MIT, AT\&T Bell Labs and General Electric worked on
        Multics. Now the Multics was formless and unfinished, darkness was over
        the surface of its shell and the spirit of Unix was hovering over the
        codebase.

        And Bell Labs said, ``Let us withdraw'' as they were not satisfied with
        the progress. And on the third day an employee of Bell Labs, Ken
        Thompson and co-worker of his, Dennis Ritchie discovered a PDP-7
        minicomputer. And they saw that the PDP was good and they decided to
        try again.
        
        \subsection{To the point}
        You ever seen \emph{The C Programming Language}? The book I mean.
        Remember who wrote it? Now take a peek at second paragraph of previous
        subsection. Rings a bell?

        First Unix was created at AT\&T Bell Labs in 1969. It was written in
        assembly language for a PDP-7 machine and soon was ported to PDP-11.
        It was a simplification of a project that K\&R worked on before
        -~Multics.

        Unix had a hierarchical file system \emph{(take that 1983 MS-DOS 2.0)},
        an assembler, a text editor and a shell. In 1973 Version 4 Unix was
        rewritten in C, an imperative high-level programming language created
        one year earlier by Dennis Ritchie and Ken Thompson for the purpose of
        making working with Unix easier, and the system itself more portable.

        Since Bell Labs was forbidden from entering any other business than
        communications services due to a 1956 decree in an antitrust case of its
        parent organization, Unix code was distributed to universities and other
        research entities\footnote{so called \emph{Research Unix}}.

        One of the places where this code landed was the Computer Systems
        Research Group at the university of California, Berkeley. Folk there
        worked hard on Unix, adding a lot of new stuff such as the {\tt vi} text
        editor, {\tt curses} library or integrating TCP/IP stack into it.

        This resulted in divergence between the AT\&T and Berkeley unices, the
        latter known as the Berkeley Software Distribution, or BSD for short.
        This funny little series of events led to a silly little thingy called
        {\bf Unix wars}, where AT\&T, Microsoft, IBM and BSD all fought over who
        has the best Unix and whose Unix should become the standard. The
        contenders were:
        \begin{itemize}
                \item UNIX System V by AT\&T
                \item Xenix by Microsoft
                \item PC/IX by IBM
                \item 4.2BSD by Berkeley
        \end{itemize}

        This is actually really fun because there will be one more unix-related
        war we will describe soon.
        Meanwhile: the result of the Unix wars was that AT\&T sued Berkeley,
        they won and BSD had to be cleansed of proprietary AT\&T code. And thus
        4.4BSD-Lite was created. Remember this one, we will get back to it soon.
\end{document}
