\documentclass[../ops.tex]{subfiles}

\begin{document}
        \subsection{What is a modern computer?}
        Computer is a synchronous, sequential, digital, electronic machine. Wow,
        that's a lot of words, what they mean tho? tldr:
        \begin{enumerate}
                \item {\bf electronic} means that it works on electricity
                \item {\bf digital} means that it interprets ranges of voltage
                        as logical states. In other words if you give it idk at
                        least 3.3 volts it treats it as logical true or a 1, it
                        does not care about the value itself.
                \item {\bf sequential} means that it has capability of
                        remembering results of what was done previously and
                        based on that it may change its behaviour.
                \item {\bf synchronous} means that it has a clock somewhere
                        inside and all its actions are conducted in-tempo with
                        this clock.
        \end{enumerate}
        It's all such a massive simplification that my digital electronics prof
        would kill me if he read this, don't show it to him. It gets the point
        across tho.

        In practice a computer is a collection of different hardware pieces that
        work together to perform certain operations and talk with the user. Each
        of them is so ridiculously complex that no one person understands them
        all on a truly deep level. Let's go through some of them in a relaxed,
        short fashion just to make sure we at least know what they do.

        \subsection{Computer memory}
        Memory, as name suggests, is used by the puter to 'member stuff. Memory
        can be either volatile or non-volatile. Volatile memory is the one that
        gets cleared when unplugged, while non-volatile does not need
        electricity to preserve its contents.

        Besides that another two important factors when characterizing memories
        are their capacity and their read and write speed (unless the memory
        is read-only, then of course it does not have a write speed).

        Computer memory follows a hierarchy dictated by price of different
        memory classes. First there is internal memory of the CPU, we will
        tackle it in a second for now it suffices to say that it is insanely
        fast but also very expensive and there's not much of it.

        Next we have {\bf primary storage}, commonly known as RAM. It's still
        relatively fast, it is volatile, users may generally change it, and put
        more of it and it is usually around few to few tens of gibibytes in
        size on personal computers.

        Next there is {\bf secondary storage}, usually either an HDD or a solid
        state drive. SSDs are quicker but more expensive and less resistant to
        frequent reads and writes in the same place. Nowadays it is usually
        around 1-2TiB in size on personal computers. It is slower than RAM, but
        non-volatile.

        There is also tertiary storage such as tape libraries or optical
        jukeboxes which are even bigger and even slower but chances of us ever
        interacting with one are relatively low so ill skip on that.
\end{document}
