\documentclass[../ops.tex]{subfiles}

\begin{document}
        \subsection{What is a modern computer?}
        Computer is a synchronous, sequential, digital, electronic machine. Wow,
        that's a lot of words, what they mean tho? tldr:
        \begin{enumerate}
                \item {\bf electronic} means that it works on electricity
                \item {\bf digital} means that it interprets ranges of voltage
                        as logical states. In other words if you give it idk
                        5 volts it treats it as logical true or a 1, it does not
                        care about the value itself.
                \item {\bf sequential} means that it has capability of
                        remembering results of what was done previously and
                        based on that it may change its behaviour.
                \item {\bf synchronous} means that it has a clock somewhere
                        inside and all its actions are conducted in-tempo with
                        this clock.
        \end{enumerate}
        It's all such a massive simplification that my digital electronics prof
        would kill me if he read this, don't show it to him. It gets the point
        across tho.

        In practice a computer is a collection of different hardware pieces that
        work together to perform certain operations and talk with the user.
        Let's go through some of them.
\end{document}
